\documentclass[a4paper,10pt]{article}
\usepackage[utf8]{inputenc}

%opening
\title{Casos de uso}
\author{Mónica Pérez Sueiro	}

\begin{document}

\maketitle

\section{Técnico de oficina:} É o encargado de xestionar as inspeccións, o tramo o día e o técnico que as vai realizar, así 
como de cambialas se fose necesario:
\begin{itemize}
 \item Crear inspección.
 \item Eliminar inspección.
 \item Asignar inspección de tramo a técnico de campo: asígnalle a un técnico específico un tramo dunha vía para unha data concreta.
 \item Desasignar inspección de tramo a técnico de campo: cambia o técnico de campo asignado a unha inspección.
 \item Responder incidencia cidadá.
 \item Modificar incidencia cidadá.
 \
\end{itemize}

\section{Técnico de campo:} Encargado de realizar inspeccións visuais nos tramos de vía que se lle asignan. 
\begin{itemize}
 \item Autenticación: cada técnico ten un usuario e un contrasinal para acceder a aplicación Android.
 \item Ver calendario do día: cada día na pantalla de inicio aparecen as inspeccións programadas para ese día.
  No mesmo caso de uso temos:
    \begin{itemize}
     \item Ver calendario de inspeccións pasadas.
     \item Ver calendario de inspeccións futuras.
    \end{itemize}
 \item Volver a día actual.
 \item Seleccionar inspección.
 \item Abrir inspección: descárganse todos os datos necesarios para a inspección ao dispositivo non que vaia traballar o técnico (ábrese o mapa).
 \item Reabrir inspección.
 \item Volver a inspección aberta.
 \item Volver ao calendario de hoxe (dende unha inspección aberta-mapa).
 \item Ampliar mapa.
 \item Reducir mapa.
 \item Crear nova deficiencia (click nun punto do mapa no que non hai elemento).
 \item Abrir elemento (click sobre elemento no mapa).
 \Item Ir a localización GPS actual.
 \item Abrir deficiencia: dende o mapa ou dende elemento.
 \item Insertar dato de deficiencia.
 \item Modificar dato de deficiencia.
 \item Eliminar dato de deficiencia.
 \item Votar ao mapa.
 \item Pechar insección: os datos tomados gardaranse en local e poden ser modificados polo técnico que realizou a inspección.
 \item Commit dos datos dunha inpección: os datos gárdanse de forma definitiva e o técnico non pode modificalos.
\end{itemize}


\section{Admin:}
\begin{itemize}
 \item Crear usuario.
 \item Eliminar usuario.
 \item Modificar datos de usuario.
 \item Asignar zona xeográfica aos técnicos: os técnicos de oficina soamente poderán xestionar as inspeccións das zonas nas que están autorizados.
 \item Deasignar zona xeográfica aos técnicos:
 \end{itemize}



\section{Xestor:}Encárgase de modificar datos cando hai cambios posteriores as inspeccións (como por exemplo que se amañou unha deficiencia), 
e tamén leva as incidencias cidadás, respondendo aos usuarios que as notificaron, modificando o seu estado, etc
\begin{itemize} 
 \item Seleccionar inspección.
 \item Abrir inspección.
 \item Modificar inspección.

\end{itemize}



\section{Usuario anónimo:}Notifican desperfectos que atopan nas vías, pero soamente aportan unha descripción.
\begin{itemize}
 \item Notificar desperfectos.
\end{itemize}


\end{document}
